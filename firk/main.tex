% main.tex
\documentclass[a4paper,9pt]{article}
\usepackage{amsmath,amssymb,amsthm,amsbsy,amsfonts}
\usepackage{todonotes}
\usepackage{systeme}
\usepackage{physics}
\usepackage{cleveref}
\newcommand{\correspondsto}{\;\widehat{=}\;}
\usepackage{bm}
\usepackage{enumitem} % label enumerate
\newtheorem{theorem}{Theorem}
\theoremstyle{definition}
\newtheorem{definition}{Definition}[section]
\theoremstyle{remark}
\newtheorem*{remark}{Remark}
% change Q.D.E symbol
\renewcommand\qedsymbol{$\hfill \mbox{\raggedright \rule{0.1in}{0.2in}}$}

\begin{document}

\author{Yingbo Ma\\
        \tt{mayingbo5@gmail.com}}
\title{Efficient Implicit Runge-Kutta Implementation}
\date{April, 2020}

\maketitle

% \section{Notation}
% \todo[inline]{Is this necessary?}

\section{Runge-Kutta Methods}
Runge-Kutta methods can numerically solve differential-algebraic equations
(DAEs) that are written in the form of
\begin{equation}
  M \dv{u}{t} = f(u, t),\quad u(a)=u_a \in \mathbb{R}^m, \quad t\in [a, b].
\end{equation}

\begin{definition} \label{def:rk}
  An \emph{$s$-stage Runge Kutta} has the coefficients $a_{ij}, b_i,$ and $c_j$
  for $i=1,2,\dots,s$ and $j=1,2,\dots,s$. One can also denote the coefficients
  simply by $\bm{A}, \bm{b},$ and $\bm{c}$. A step of the method is
  \begin{equation} \label{eq:rk_sum}
    u_{n+1} = u_n + \sum_{i=1}^s b_i k_i,
  \end{equation}
  where
  \begin{align} \label{eq:rk_lin}
    M z_i &= \sum_{j=1}^s a_{ij}k_j\qq{or} (I_s \otimes M) \bm{z} =
    (\bm{A}\otimes I_m)\bm{k}, \quad g_i = u_n + z_i \\
    k_i &= hf(g_i, t+c_ih).
  \end{align}
\end{definition}

We observe that when $a_{ij} = 0$ for each $j\ge i$, \cref{eq:rk_lin} can be
computed without solving a system of algebraic equations, and we call methods
with this property \emph{explicit} otherwise \emph{implicit}.

We should note solving $z_i$ is much preferred over $g_i$, as they have a
smaller magnitude. A method is stiffly accurate, i.e. the stability function
$R(\infty) = 0$, when the matrix $\bm{A}$ is fully ranked and $a_{si} = b_i$.
Hence, for such methods $\bm{b}^T\bm{A}^{-1}$ gives the last row of the identity
matrix $I_s$. We can use this condition to further simplify \cref{eq:rk_sum} (by
slight abuse of notation):
\begin{equation} \label{eq:rk_sim}
  u_{n+1} = u_n + \bm{b}^T\bm{k} = u_n +
  \bm{b}^T\underbrace{\bm{A}^{-1}\bm{z}}_\text{\cref{eq:rk_lin}} = u_n +
  \mqty(0 & \cdots & 0 & 1)\bm{z} = u_n + z_s.
\end{equation}

\section{Solving Nonlinear Systems from Implicit Runge-Kutta Methods}
We have to solve \cref{eq:rk_lin} for $\bm{z}$ when a Runge-Kutta method is
implicit. More explicitly, we need to solve $G(\bm{z}) = \bm{0}$, where
\begin{equation} \label{eq:nonlinear_rk}
  G(\bm{z}) = (I_s \otimes M) \bm{z} - h(\bm{A}\otimes I_m) \tilde{\bm{f}}(\bm{z}) \qq{and}
  \tilde{f}(\bm{z})_i = f(u_n+z_i, t+c_i h)
\end{equation}
The propose of introducing a computationally expensive nonlinear system solving
step is to combat extreme stiffness. The Jacobian matrix arising from
\cref{eq:rk_lin} is ill-conditioned due to stiffness. Thus, we must use Newton
iteration to ensure stability, since fixed-point iteration only converges for
contracting maps, which greatly limits the step size.

Astute readers may notice that the Jacobian
\begin{equation}
  \tilde{\bm{J}}_{ij} = \pdv{\tilde{\bm{f}}_i}{\bm{z}_j} = \pdv{f(u_n + z_i, t+c_ih)}{u}
\end{equation}
requires us to compute the Jacobian of $f$ at $s$ many points, which can very
expensive. We can approximate it by
\begin{equation}
  \tilde{\bm{J}}_{ij} \approx J = \pdv{f(u_n, t)}{u}.
\end{equation}
Our simplified Newton iteration from \cref{eq:nonlinear_rk} is then
\begin{align} \label{eq:newton_1}
  (I_s \otimes M - h\bm{A}\otimes J) \Delta \bm{z}^k &= -G(\bm{z}^k) = -(I_s
  \otimes M) \bm{z}^k + h(\bm{A}\otimes I_m) \tilde{\bm{f}}(\bm{z}^k) \\
  \bm{z}^{k+1} &= \bm{z}^{k} + \Delta \bm{z}^k \label{eq:newton_2},
\end{align}
where $\bm{z}^k$ is the approximation to $\bm{z}$ at the $k$-th iteration.

\subsection{Change of Basis}
\todo[inline]{discuss SDIRK}In Hairer's Radau IIA implementation, he left
multiplies \cref{eq:newton_1} by $(hA)^{-1} \otimes I_m$ to exploit the
structure of the iteration matrix \cite{hairer1999stiff}, so we have
\begin{align}
  ((hA)^{-1} \otimes I_m)(I_s \otimes M) &= (hA)^{-1} \otimes M \\
  ((hA)^{-1} \otimes I_m)(h\bm{A}\otimes J) &= I_s\otimes J \\
  ((hA)^{-1} \otimes I_m)G(\bm{z}^k) &= ((hA)^{-1} \otimes M) \bm{z}^k -
  \tilde{\bm{f}}(\bm{z}^k),
\end{align}
and finally,
\begin{equation} \label{eq:newton1}
  ((hA)^{-1} \otimes M - I_s\otimes J) \Delta \bm{z}^k = -((hA)^{-1} \otimes M)
  \bm{z}^k + \tilde{\bm{f}}(\bm{z}^k).
\end{equation}
Hairer also diagonalizes $A^{-1}$, i.e.
\begin{equation}
  V^{-1}A^{-1}V = \Lambda,
\end{equation}
to decouple the $sm \times sm$ system. To transform \cref{eq:newton1} to the
eigenbasis of $A^{-1}$, notice
\begin{equation}
  A^{-1}x = b \implies V^{-1}A^{-1}x = V^{-1}b \implies \Lambda V^{-1}x =
  V^{-1}b.
\end{equation}
Similarly, we have
\begin{align}
  &(h^{-1} \Lambda \otimes M - I_s\otimes J) (V^{-1}\otimes I_m)\Delta\bm{z}^k\\
  =& (V^{-1}\otimes I_m)(-((hA)^{-1} \otimes M)
  \bm{z}^k + \tilde{\bm{f}}(\bm{z}^k)).
\end{align}
We can introduce the transformed variable $\bm{w} = (V^{-1}\otimes I_m) \bm{z}$
to further reduce computation, so \cref{eq:newton_1} and \cref{eq:newton_2} is now
\begin{align} \label{eq:newton2}
  (h^{-1} \Lambda \otimes M - I_s\otimes J) \Delta\bm{w}^k
  &= -(h^{-1} \Lambda \otimes M) \bm{w}^k +
  (V^{-1}\otimes I_m)\tilde{\bm{f}}((V\otimes I_m)\bm{w}^k) \\
  \bm{w}^{k+1} &= \bm{w}^{k} + \Delta \bm{w}^k.
\end{align}
People usually call the matrix $W=(h^{-1} \Lambda \otimes M - I_s\otimes J)$ the
iteration matrix or the $W$-matrix.

\subsection{Stopping Criteria}
Note that throughout this subsection, $\norm{\cdot}$ denotes the norm that is
used by the time-stepping error estimate. By doing so, we can be confident that
convergent results from a nonlinear solver do not introduce step rejections.

There are two approaches to estimate the error of a nonlinear solver, by the
displacement $\Delta \bm{z}^k$ or by the residual $G(\bm{z}^k)$. The residual
behaves like the error scaled by the Lipschitz constant of $G$. Stiff equations
have a large Lipschitz constant, furthermore, this constant is not known a
priori. This makes the residual test unreliable. Hence, we are going to focus on
the analysis of the displacement.

Simplified Newton iteration converges linearly, so we can model the convergence
process as
\begin{equation}
  \norm{\Delta \bm{z}^{k+1}} \le \theta \norm{\Delta \bm{z}^{k}}.
\end{equation}
The convergence rate at $k$-th iteration $\theta_k$ can be estimated by
\begin{equation}
  \theta_k = \frac{\norm{\Delta\bm{z}^{k}}}{\norm{\Delta\bm{z}^{k-1}}},\quad k\ge 1.
\end{equation}
Notice we have the relation
\begin{equation}
  \bm{z}^{k+1} - \bm{z} = \sum_{i=0}^\infty \Delta\bm{z}^{k+i+1}.
\end{equation}
If $\theta<1$, by the triangle inequality, we then have
\begin{equation}
  \norm{\bm{z}^{k+1} - \bm{z}} \le
  \norm{\Delta\bm{z}^{k+1}}\sum_{i=0}^\infty \theta^i \le \theta
  \norm{\Delta\bm{z}^{k}}\sum_{i=0}^\infty \theta^i = \frac{\theta}{1-\theta}
  \norm{\Delta\bm{z}^{k}}.
\end{equation}
To ensure the nonlinear solver error does not cause step rejections, we need a
safety factor $\kappa = 1/10$. Our first convergence criterion is
\begin{equation}
  \eta_k \norm{\Delta\bm{z}^k} \le \kappa, \qq{if}
  k \ge 1 \text{ and } \theta \le 1, ~ \eta_k=\frac{\theta_k}{1-\theta_k}.
\end{equation}
One major drawback with this criterion is that we can only check it after one
iteration. To cover the case of convergence in the first iteration, we need to
define $\eta_0$. It is reasonable to believe that the convergence rate remains
relatively constant with the same $W$-matrix, so if $W$ is reused
\todo[inline]{Add the reuse logic section} from the previous nonlinear solve,
then we define
\begin{equation}
  \eta_0 = \eta_{\text{old}},
\end{equation}
where $\eta_{\text{old}}$ is the finial $\eta_k$ from the previous nonlinear
solve, otherwise we define
\begin{equation}
  \eta_0 = \max(\eta_{\text{old}}, ~\text{eps}(\text{relative
  tolerance}))^{0.8}.
\end{equation}
In the first iteration, we also check
\begin{equation}
  \Delta\bm{z}^{1} < 10^{-5}
\end{equation}
for convergence. \todo[inline]{OrdinaryDiffEq.jl also checks
\texttt{iszero(ndz)}. It seems redundant in hindsight.}

Moreover, we need to define the divergence criteria. It is obvious that we want
to limit $\theta$ to be small. Therefore, the first divergence criterion is
\begin{equation}
  \theta_k > 2.
\end{equation}
Also, the algorithm diverges if the max number of iterations $k_{\max}$ is
reached without convergence. A subtler criterion for divergence is: no
convergence is predicted by extrapolating to $\norm{\Delta\bm{z}^k_{\max}}$,
i.e.
\begin{equation}
  \frac{\theta_k^{k_{\max}-k}}{1-\theta_k} \norm{\Delta\bm{z}^k} > \kappa.
\end{equation}
\todo[inline]{OrdinaryDiffEq.jl doesn't actually check this condition anymore.}

\subsection{$W$-matrix Reuse}

\section{Step Size Control}
\subsection{Smooth Error Estimation}
\subsection{Standard (Integral) Control}
\subsection{Predictive (Modified PI) Control}

\nocite{hairer2010solving}

\bibliography{reference.bib}
\bibliographystyle{siam}

\end{document}
